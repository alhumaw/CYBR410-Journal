% Options for packages loaded elsewhere
\PassOptionsToPackage{unicode}{hyperref}
\PassOptionsToPackage{hyphens}{url}
%
\documentclass[
]{article}
\usepackage{amsmath,amssymb}
\usepackage{iftex}
\ifPDFTeX
  \usepackage[T1]{fontenc}
  \usepackage[utf8]{inputenc}
  \usepackage{textcomp} % provide euro and other symbols
\else % if luatex or xetex
  \usepackage{unicode-math} % this also loads fontspec
  \defaultfontfeatures{Scale=MatchLowercase}
  \defaultfontfeatures[\rmfamily]{Ligatures=TeX,Scale=1}
\fi
\usepackage{lmodern}
\ifPDFTeX\else
  % xetex/luatex font selection
\fi
% Use upquote if available, for straight quotes in verbatim environments
\IfFileExists{upquote.sty}{\usepackage{upquote}}{}
\IfFileExists{microtype.sty}{% use microtype if available
  \usepackage[]{microtype}
  \UseMicrotypeSet[protrusion]{basicmath} % disable protrusion for tt fonts
}{}
\makeatletter
\@ifundefined{KOMAClassName}{% if non-KOMA class
  \IfFileExists{parskip.sty}{%
    \usepackage{parskip}
  }{% else
    \setlength{\parindent}{0pt}
    \setlength{\parskip}{6pt plus 2pt minus 1pt}}
}{% if KOMA class
  \KOMAoptions{parskip=half}}
\makeatother
\usepackage{xcolor}
\ifLuaTeX
  \usepackage{luacolor}
  \usepackage[soul]{lua-ul}
\else
  \usepackage{soul}
\fi
\setlength{\emergencystretch}{3em} % prevent overfull lines
\providecommand{\tightlist}{%
  \setlength{\itemsep}{0pt}\setlength{\parskip}{0pt}}
\setcounter{secnumdepth}{-\maxdimen} % remove section numbering
\ifLuaTeX
  \usepackage{selnolig}  % disable illegal ligatures
\fi
\usepackage{bookmark}
\IfFileExists{xurl.sty}{\usepackage{xurl}}{} % add URL line breaks if available
\urlstyle{same}
\hypersetup{
  pdftitle={Lab 4 - More k8s},
  hidelinks,
  pdfcreator={LaTeX via pandoc}}

\title{Lab 4 - More k8s}
\author{}
\date{}

\begin{document}
\maketitle

\subsection{Cybersecurity}\label{cybersecurity}

\section{Lab 4 - More k8s}\label{lab-4---more-k8s}

CYBR410 - Applied Cyber Operations

\st{04-29-24 - Alexander Moomaw}

\subsection{Simple Container}\label{simple-container}

\emph{Create a simple program that we can easily tell is running behind
a load balancer. This can be as easy as creating a random number each
time the program starts (that is returned to the user) or a counter that
increments each time the page is loaded.}

\subsubsection{Create the server:}\label{create-the-server}

\begin{itemize}
\tightlist
\item
  Utilize Pythons\textquotesingle{} \emph{flask} and \emph{random}
  module
\item
  Generate a global random int between 1 and 100
\item
  Create a function bound to our web servers root directory that serves
  our index.html file and inserts our random integer into the home page
\item
  Serve our content locally on port 8000
\end{itemize}

\subsubsection{Inside index.html:}\label{inside-index.html}

\begin{itemize}
\tightlist
\item
  Make a new directory inside our web server directory, named templates.
\item
  Place our index.html file in the templates directory
\item
  Create generic content that displays that random integer
\end{itemize}

\subsection{Kubernetes}\label{kubernetes}

\emph{As before, push the new container to our k8s registry and create
the two yaml files needed to run your container in k8s. Make sure you
can reach your service and that it's running via a k8s pod (exactly the
same thing as last lab).}

\subsubsection{To push our newly created web server to
kubernetes:}\label{to-push-our-newly-created-web-server-to-kubernetes}

\begin{itemize}
\tightlist
\item
  \texttt{docker\ tag\ flask\ localhost:32000/flask:k8s}
\item
  \texttt{docker\ push\ localhost:32000/flask:k8s}
\end{itemize}

\subsubsection{What needs to change in our YAML
file:}\label{what-needs-to-change-in-our-yaml-file}

\begin{itemize}
\tightlist
\item
  Inside test.yaml:

  \begin{itemize}
  \tightlist
  \item
    \texttt{image:\ localhost:32000/flask:k8s}
  \end{itemize}
\end{itemize}

\subsubsection{Spin up our cluster:}\label{spin-up-our-cluster}

\begin{itemize}
\tightlist
\item
  \texttt{microk8s.kubectl\ apply\ -f\ test.yaml}
\item
  \texttt{microk8s.kubectl\ apply\ -f\ test-svc.yaml}
\end{itemize}

\subsection{More Kubernetes}\label{more-kubernetes}

\emph{Delete a process running in the pod/container you created (kill -9
PID). Run the command that shows the pods (microk8s.kubectl get pods
--all-namespaces), has anything changed? Elaborate.}

\emph{Try and reach the service you created many times (curl or
browser). Can you observe the load balancer at work? Do you always get
the same page? Do you always get a different page?}

\emph{Describe a service (microk8s.kubectl describe service SERVICE
NAME).Can I reach each individual pod being load balanced? Where do you
see this information? Explain how this could be useful}

\end{document}
